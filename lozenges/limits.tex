\documentclass{article}
\usepackage{amsmath,amssymb}
\begin{document}
We calculate the GHY boundary term for the flat causal diamond in $D$ dimensions via two limiting procedures. Take standard polar coordinates $(t,r,\boldsymbol\theta)$. The tips of the diamond are at $(\pm T,\mathbf0)$. Consider two $\epsilon$-families of surfaces:
\begin{itemize}
\item Timelike lozenges (TL): 	$$t^2 - (T-r)^2 + \epsilon^2 = 0$$
\item Spacelike lozenges  (SL):	$$r^2 - (T-|t|)^2 + \epsilon^2  = 0$$
\end{itemize}
both approaching the diamond as $\epsilon\rightarrow0$. If we calculate the boundary action associated with these surfaces for non-zero $\epsilon$ and take the limit of that expression as $\epsilon\rightarrow0$ we obtain the following results. For the TL family we get
$$
S_{GHY} \sim2 V_{D-2}T^{D-2}\left[2(D-2)-\log\left(\frac{2T}{\epsilon}\right)\right]
$$
and for the SL family we obtain
$$
S_{GHY} \sim
\begin{cases}
\displaystyle4\log\left(\frac{2T}{\epsilon}\right) \qquad&\text{for }D=2\\
\displaystyle2 V_{D-2}T^{D-2}\frac{D-1}{D-2}\qquad&\text{for }D>2
\end{cases}
$$
where $V_n$ is the volume of the $n$-sphere. So the limits agree for $D=2$ but disagree for $D>2$.  The SL limit gives a finite constant boundary term for $D>2$ whereas the TL limit gives a logarithmic divergence in all $D$.

\end{document}